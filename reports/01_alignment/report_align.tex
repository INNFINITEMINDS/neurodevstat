%%% LaTeX Template
%%% This template can be used for both articles and reports.
%%%
%%% Copyright: http://www.howtotex.com/
%%% Date: February 2011

%%% Preamble
\documentclass[paper=a4, fontsize=11pt]{scrartcl} % Article class of KOMA-script with 11pt font and a4 format

\usepackage[english]{babel} % English language/hyphenation
\usepackage[protrusion=true,expansion=true]{microtype} % Better typography
\usepackage{amsmath,amsfonts,amsthm} % Math packages
\usepackage[pdftex]{graphicx} % Enable pdflatex
%\usepackage{color,transparent} % If you use color and/or transparency
\usepackage[hang, small,labelfont=bf,up,textfont=it,up]{caption} % Custom captions under/above floats
\usepackage{epstopdf} % Converts .eps to .pdf
\usepackage{subfig} % Subfigures
\usepackage{booktabs} % Nicer tables
\usepackage{hyperref}

%%% Advanced verbatim environment
\usepackage{verbatim}
\usepackage{fancyvrb}
\DefineShortVerb{\|} % delimiter to display inline verbatim text


%%% Custom sectioning (sectsty package)
\usepackage{sectsty} % Custom sectioning (see below)
\allsectionsfont{ % Change font of al section commands
  \usefont{OT1}{bch}{b}{n} % bch-b-n: CharterBT-Bold font
% \hspace{15pt} % Uncomment for indentation
}

\sectionfont{ % Change font of \section command
 \usefont{OT1}{bch}{b}{n} % bch-b-n: CharterBT-Bold font
 \sectionrule{0pt}{0pt}{-5pt}{0.8pt} % Horizontal rule below section
}


%%% Custom headers/footers (fancyhdr package)
\usepackage{fancyhdr}
\pagestyle{fancyplain}
\fancyhead{} % No page header
%\fancyfoot[C]{\thepage} % Pagenumbering at center of footer
%\fancyfoot[R]{\small \texttt{}} % You can remove/edit this line 
\renewcommand{\headrulewidth}{0pt} % Remove header underlines
\renewcommand{\footrulewidth}{0pt} % Remove footer underlines
\setlength{\headheight}{12.0pt}

%%% Equation and float numbering
%\numberwithin{equation}{section}
% Equationnumbering: section.eq#
%\numberwithin{figure}{section}
% Figurenumbering: section.fig#
%\numberwithin{table}{section}
% Tablenumbering: section.tab#


%%% Title
\title{ \vspace{-1in} \usefont{OT1}{bch}{b}{n}
\huge \strut Pseudo-Alignment with
{\href{https://pachterlab.github.io/kallisto/about}{kallisto}} \strut \\
\Large \bfseries \strut Efficient Probabilistic Quantification of RNA-Seq Reads
  \strut
}
\author{ \usefont{OT1}{bch}{m}{n}
        Nima Hejazi\\ \usefont{OT1}{bch}{m}{n}
        Division of Biostatistics\\
        University of California, Berkeley\\ \usefont{OT1}{bch}{m}{n}
        \texttt{nh@nimahejazi.org}
}
\date{}

%%% Begin document
\begin{document}
\maketitle
\section{Pseudo-Alignment of RNA-Seq Reads}
The present project concerns the re-analysis of transcriptomic data from the
study described in the paper
{\href{http://www.nature.com/neuro/journal/v18/n1/abs/nn.3898.html}{``Developmental
regulation of human cortex transcription and its clinical relevance at base
resolution''}}, Jaffe \textit{et al.}, \textit{Nature Neuroscience}. In order to
quantify RNA-Seq reads, alignment against a reference transcriptome must be
performed, a procedure which results in tables of read counts for use in
downstream statistical analysis. Here, we take advantage of
\textbf{pseudo-alignment}, a novel development in sequencing algorithms, to
probabilistically align reads. Below, we describe pseudo-alignment and the
results of its application to the Jaffe \textit{et al.} data.

\subsection{The Pseudo-Alignment Process}
Pseudo-alignment is a novel process for quantifying a set of samples of RNA-Seq
reads by performing partial matching against a reference transcriptome. The
novel pseudo-alignment process, implemented in the command line tool
\textbf{kallisto}, takes into account all of the information contained in a set
of reads while reducing the computational burden imposed by more traditional
alignment techniques. The \textbf{kallisto} tool provides results similar to
that produced by other alignment software (\textit{e.g.}, \textbf{bowtie}),
while taking only a fraction of the time. For a complete description of
pseudo-alignment, consult the paper
{\href{http://www.nature.com/nbt/journal/v34/n5/full/nbt.3519.html}{``Near-optimal
probabilistic RNA-seq quantification''}}, Bray \textit{et al.}, \textit{Nature
Biotechnology}.

\subsection{The Results of Pseudo-Alignment}
The pseudo-alignment procedure was implemented on the Jaffe \textit{et al.} data
through the use of the \textbf{kallisto} command line tool. Using a publicly
available trascriptome assembled from the GRCh38 (hg19) \textit{Homo sapiens}
genome, sets of paired-end RNA-Seq reads for each of the 12 subjects involved in
the study were pseudo-aligned, resulting in count tables mapping each set of
reads to \textbf{173,259} transcriptomic objects. \textit{Please note that while
similar quantification tools (e.g., Cufflinks) produce estimates of mappings of
isoforms, kallisto provides estimates of transcript abundance}. Though this is
the first time that I have used \textbf{kallisto} for alignment and RNA-Seq
quantification, elementary searches for the ``run\textunderscore info.json''
output file (from other unrelated projects) seem to suggest that the produced
transcript abundances are normal for \textbf{kallisto}. Tables of counts are
produced for each set of paired-end RNA-Seq reads for each subject in a
tab-separated file format; these files are suitable for concatenation into a
single count table containing quantification results for all subjects, which can
be used as input to a set of statistical analysis scripts after appropriate data
cleaning.

\end{document}
